Anderson and Moore \cite{ANDER:AIM2} outlines a procedure that computes solutions for structural models of the form
\begin{gather}
\sum_{i= - \tau}^\theta { H_i ~ x_{ t + i } }= 0~,~ t \geq 0 \label{eq:canonical}\\ \intertext{ with initial conditions, if any, given by constraints of the form}
x_t ~ = ~ x^{data}_t ~ , ~ t = ~ - \tau~, \ldots ,~ -1\\ \intertext{where both $\tau$ and $\theta$ are non-negative, and $x_t$ is an L dimensional vector with}
\lim_{ t \rightarrow \infty } x_t ~ = ~ 0
\end{gather}

The algorithm determines whether
the model \ref{eq:canonical} has a unique solution, an infinity of
 solutions or no solutions at all.

The specification \ref{eq:canonical} is not restrictive.
One can handle inhomogeneous version  of equation \ref{eq:canonical}
by recasting the problem in terms of  deviations from a steady state value or
by adding a new variable for each non-zero right hand side with an equation
guaranteeing the  value always equals  
the inhomogeneous value($x^{con}_t =x^{con}_{t-1}$ and $x^{con}_{t-1} = x^{RHS}$).



Saddle point problems combine initial conditions and asymptotic 
convergence to identify their solutions.
The uniqueness of solutions to 
system  \ref{eq:canonical} requires that
the transition matrix characterizing the linear system have an appropriate
number of explosive and stable eigenvalues\cite{blanchard80},
and that the asymptotic linear constraints 
are linearly independent of explicit and implicit initial 
conditions\cite{ANDER:AIM2}.

The solution methodology entails 
\begin{enumerate}
\item using equation \ref{eq:canonical} to
compute a state space transition matrix.
\item Computinging the eigenvalues and the invariant space associated with
large eigenvalues
\item Combining the constraints provided by:
  \begin{enumerate}
  \item the
initial conditions,
\item  auxiliary initial conditions identified in the computation of the transisiton matrix and 
\item the invariant space vectors
  \end{enumerate}
\end{enumerate}

Figure \ref{fig:overview} presents a flow chart  summarizing the
algorithm. 
%For a description of a parallel implementation see \cite{ANDER:PARA}
%For a description of a continuous application see \cite{anderson97}.

\begin{figure}[htbp]
  \begin{center}
\includegraphics[width=8cm]{overallGraph.eps}
%   \caption{Algorithm Overview}
    \label{fig:overview}
  \end{center}
\end{figure}

%Anderson and Moore \cite{ANDER:AIM2} demonstrates that 



%Given the coefficient matrix
%\begin{gather*}
%\matob{H_{-\tau}}{H_\theta}
%\end{gather*}
%the procedure computes the reduced form coefficient matrix
%\begin{gather*}
%\matob{B_{-\tau}}{ B_{-1}}
%\end{gather*}
%for any model satisfying assumptions \ref{asmone} and \ref{asmtwo}.  If the model does not satisfy assumptions \ref{asmone} and \ref{asmtwo}, the procedure indicates whether there are no convergent solutions or a multiplicity of convergent solutions.



  

